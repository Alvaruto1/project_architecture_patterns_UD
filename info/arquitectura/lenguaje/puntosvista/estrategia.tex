\subsection{Punto de Vista Estratégico}
Para describir los aspectos de la estrategia de nuestra empresa, se explicaran los siguientes puntos de vista presentado diferentes perspectivas de la aplicación del modelo a la dirección de nuestra empresa:
\subsubsection{Punto de vista de estrategia}
\begin{table}[th!]
	\begin{center}
		\begin{tabular}{| c | p{6cm} |} %r c|}
			\hline
			\multicolumn{2}{ |c| }{Punto de vista de la estrategia}
			
			\\ \hline
			
			Stakeholders
			& 
			CxOs(Chief eXperience Officer), gerentes del negocio y arquitectos empresariales y de negocio. 
			
			\\ \hline
			Ocupaciones 
			& 
			Desarrollador de estrategias.
			
			\\ \hline
			
			Propósito 
			& 
			Desarrollo y toma de decisiones.
			
			\\ \hline
			
			Alcance 
			& 
			Estrategia. 
			
			\\ \hline
			
			Elementos relacionados 
			& 
			Cursos de acción, capacidad, flujo del valor, recurso, Outcomes.     		
			
			\\ \hline
		\end{tabular}
		\caption{Descripción del punto de vista de estrategia}
	\end{center}
\end{table}

\subsubsection{Punto de vista del mapeado de las capacidades}
\par Permite mostrar un punto de vista general de las capacidades de la empresa; típicamente el mapeado nos muestra 2 o 3 niveles de las capacidades de una empresa, por ejemplo se puede crear un mapeado de calor que nos muestre áreas de inversión, además en algunos casos nos permite especificar las salidas con sus determinadas capacidades.
\begin{table}[th!]
	\begin{center}
		\begin{tabular}{| c | p{8cm} |} %r c|}
			\hline
			\multicolumn{2}{ |c| }{mapeado de las capacidades}
			
			\\ \hline
			
			Stakeholders
			& 
			Gerentes del negocio y arquitectos empresariales y de negocio. 
			
			\\ \hline
			Ocupaciones 
			& 
			Arquitecto de estrategias, tácticas y motivación.
			
			\\ \hline
			
			Propósito 
			& 
			Desarrollo y toma de decisiones.
			
			\\ \hline
			
			Alcance 
			& 
			Estrategia. 
			
			\\ \hline
			
			Elementos relacionados 
			& 
			capacidad, recurso, Outcomes.     		
			
			\\ \hline
		\end{tabular}
		\caption{Descripción del punto de vista del mapeado de las capacidades}
	\end{center}
\end{table}

\subsubsection{Punto de vista del flujo del valor}
\par Permite mostrar un punto de vista general del flujo del valor; Identifica las capacidades que soportan las etapas en el flujo del valor, valor creado y stakeholders involucrados.
\begin{table}[th!]
	\begin{center}
		\begin{tabular}{| c | p{8cm} |} %r c|}
			\hline
			\multicolumn{2}{ |c| }{Flujo del valor}
			
			\\ \hline
			
			Stakeholders
			& 
			Gerentes del negocio y arquitectos empresariales y de negocio. 
			
			\\ \hline
			Ocupaciones 
			& 
			Arquitecto de estrategias, tácticas y motivación.
			
			\\ \hline
			
			Propósito 
			& 
			Desarrollo y toma de decisiones.
			
			\\ \hline
			
			Alcance 
			& 
			Estrategia. 
			
			\\ \hline
			
			Elementos relacionados 
			& 
			capacidad, flujo del valor, Outcomes, stakeholders.     		
			
			\\ \hline
		\end{tabular}
		\caption{Descripción del punto de vista del flujo del valor}
	\end{center}
\end{table}

\subsubsection{Punto de vista realización de resultados}
\par Este punto de vista es usado para mostrar el nivel más alto de la arquitectura, muestra los negocios orientados a los resultados producidos por las capacidades y los elementos adyacentes al núcleo de la empresa.
\begin{table}[th!]
	\begin{center}
		\begin{tabular}{| c | p{8cm} |} %r c|}
			\hline
			\multicolumn{2}{ |c| }{Realización de resultados}
			
			\\ \hline
			
			Stakeholders
			& 
			Gerentes del negocio y arquitectos empresariales y de negocio. 
			
			\\ \hline
			Ocupaciones 
			& 
			Negocios orientados a resultados.
			
			\\ \hline
			
			Propósito 
			& 
			Desarrollo y toma de decisiones.
			
			\\ \hline
			
			Alcance 
			& 
			Estrategia. 
			
			\\ \hline
			
			Elementos relacionados 
			& 
			capacidad, flujo del valor, recursos, Outcomes, valor, aceptación, elementos del núcleo.     		
			
			\\ \hline
		\end{tabular}
		\caption{Descripción punto de vista de realización de resultados}
	\end{center}
\end{table}

\subsubsection{Punto de vista mapeado de recursos}
\par Permite crear y estructurar una visión general de los recursos de la empresa; típicamente el mapeado nos muestra 2 o 3 niveles de las capacidades de una empresa, por ejemplo se puede crear un mapeado de calor que nos muestre áreas de inversión, además en algunos casos nos permite especificar las relaciones entre los recursos y las capacidades asignadas.
\begin{table}[th!]
	\begin{center}
		\begin{tabular}{| c | p{8cm} |} %r c|}
			\hline
			\multicolumn{2}{ |c| }{mapeado de recursos}
			
			\\ \hline
			
			Stakeholders
			& 
			Gerentes del negocio y arquitectos empresariales y de negocio. 
			
			\\ \hline
			Ocupaciones 
			& 
			Arquitecto de estrategias, tácticas y motivación.
			
			\\ \hline
			
			Propósito 
			& 
			Desarrollo y toma de decisiones.
			
			\\ \hline
			
			Alcance 
			& 
			Estrategia. 
			
			\\ \hline
			
			Elementos relacionados 
			& 
			capacidad,recursos, paquetes de trabajo.     		
			
			\\ \hline
		\end{tabular}
		\caption{Descripción punto de vista mapeado de recursos}
	\end{center}
\end{table}